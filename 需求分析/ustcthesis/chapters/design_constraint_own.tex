\chapter{总体设计约束}
\section{标准符合性}
本软件会发布在ios平台、android平台上,对应的版本有相应的标准和规范。具体内容会在技术限制中详细说明。本软件的客户端和服务器通信时采用国际标准的通信协议,包括但不限于TCP,UDP等。本软件在服务器端使用数据库存储用户数据,数据库应当符合国家标准。本软件在发布时必须符合当地及其国家的技术标准和法律。

\section{硬件约束}
\subsection{ios}
本软件的ios版本符合Apple公司的AppStore要求的规范,
%https://developer.apple.com/app-store/review/guidelines/cn/#hardware-compatibility
例如:
\begin{itemize}
  \item 通过设计,使App节省能耗。App不应快速耗尽电池电能、产生过多的热量或对设备资源造成不必要的负担。
  \item App 不得建议或要求重新启动设备。
  \item App 必须适当地沙盒化,并遵循“macOS File System Documentation”。
  \item 必须使用 Xcode 中提供的技术来进行打包和提交,不允许使用第三方安装器。
  \item 不得自动启动或者在启动时包含其他自动运行的代码,不得在未经同意的情况下登录,也不得大量生成在用户退出 app 后仍在未经同意的情况下继续运行的进程。
  \item 不得下载或安装独立的 App、kext、额外代码或资源,以向我们在审核过程中看到的 App添加功能,或进行大幅更改。
  \item 不得申请升级至 root 特权或使用 setuid 属性。
  \item 不得在启动时显示许可证屏幕、需要使用许可证密匙或实施自己的拷贝保护措施。
\end{itemize}

\subsection{Android}
本软件的Android版本应当符合Android6.0的要求规范,对于较早的Android版本可能会出现问
题。所以本软件会做一些测试,在不符合要求的的Android版本上会停止运行以避免对用户的设备造成损害。


\section{技术限制}
\item本软件将使用通用的技术,避免使用没有通用标准的技术。
\item本软件的接口都使用平台所对应的标准接口,具体使用到的接口必须在技术文档中做详细的说明。
\item本软件的数据库使用的是MySQL数据库,并使用最新的版本,应避免使用本数据库不具有的功能。
\item本软件不同用户可以并行操作,相同用户可以在不同的客户端中同时登陆服务器。当用户登录时会从服务器端同步数据,用户的所有编辑都在本地进行。用户端和服务器端只会传输完整的文件。
\item为了避免丢失数据,本软件使用TCP协议用于通信协议,所有的规范都以RFC文档为标准。
\item为了统一规范以便于后期的调试和修改,所有的开发人员必须遵循相应的设计规范,同时也 必须遵循相应的编程规范。
