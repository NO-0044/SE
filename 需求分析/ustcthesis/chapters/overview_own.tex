\chapter{总体概述}
\section{软件概述}
\subsection{项目介绍}
本项目主要目的为开发一款学习辅助系统。当前市面上已经存在很多学习辅助系统,并且已经在很多学校和在线教育平台投入使用。但是普遍存在以下这些问题:\\
\begin{itemize}
  \item 各产品功能分散不完善,往往需要同时交错使用多个系统
  \item 功能落后,更新不及时
  \item 用户界面不美观,用户体验差
  \item 处理不够智能,系统内部模块协同差,使用不便捷
\end{itemize}
为此,我们提出了此项目,希望能够整合现有各系统的各项功能并在其基础之上加强模块协同,融合新的设计理念,加强用户体验。最终能够更好的服务于广大教师的教学工作并能够为同学的学习提供辅助,提高学习效率。

\subsection{产品环境介绍}
本产品可以作为单独的产品使用,但是也可以对接学校现有的综合教务系统或者后台数据库。哦嗯是本产品提供可扩展的外部接口用于管理员自行扩展系统功能

\section{软件功能}
软件分为学生端、教师端和教学管理人员端。不同的具有不同的权限并且拥有不同的功能。本说明按照不同的功能进行组织,在具体功能时指明相应拥有权限的用户,不单独介绍各类用户功能。软件拥有的基本功能模块如下:
\begin{itemize}
  \item
\end{itemize}
\section{用户特征}
本系统用户主要有四类:学生、教师、教学管理人员,下面给出各类人员的特征描述
\subsection{学生}
学生用户需要是相应课程的注册用户,具体需要经过相应老师和教学管理人员的同意。学生为相应资源的使用者和课程参与者。同时也参与讨论与资源分享。\\
学生需要有基本的软件的软件使用能力,并且拥有相应的客户端。
\subsection{教师}
教师身份同样需要经过认证,教师可以注册生成响应的课程。教师应该拥有一定的教学经验,并且能够使用软件辅助教学,进行资源分享和作业等的布置等等。
\subsection{教学管理人员}
教学管理人员为高校负责教学学生相关的人员,主要负责协助教师安排课时以及了解学生学习进度指导完成学业。\\
教学管理人员需要能够正确了解相应课程要求以及学生的基本信息,正确管理相关信息数据库。

\section{假设和依赖关系}
项目用户管理部分需要对接高校已有的身份认证接口以核实真实身份。\\
项目主要面向Android和IOS客户端,Android客户端采用AndroidStudio,IOS客户端开发环境为XCODE。\\
