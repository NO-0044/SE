\chapter{其他需求}

\section{数据库}

用户数据初始化:在学期初从教务系统导入所有用户的课程数据,创建课程表和用户表。

用户登录:在用户登录时返回用户基本表中的信息,和其他表项如课程,讨论,笔记等信息。

用户同步:在用户进行同步或上传操作时更新数据库信息,在要求下载时传输用户需求的数据。

用户退出:在用户退出时安全关闭以保证用户信息不被修改。


\section{操作}

课表查询:用户通过对课表的查询访问数据库,查看课表。

日程查看:用户登录系统之后客户端从服务器端下载用户日程(实验,作业,课表)信息,并以日历流水形式显示在首页

创建笔记:调用本地文本编辑器,编写笔记,并上传到服务器端

成绩查询:用户访问数据库依据自己的权限查看成绩

实验作业提交:用户通过客户端上传实验报告到服务器相应空间

资料上传:教师上传课程所需的课件或实验资料到服务器端,并按需发布新的作业或任务

成绩录入:教师录入学生考试或平时作业实验成绩

\section{本地化}

描述支持多语种的需求。
