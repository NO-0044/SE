\chapter{具体需求}
  \section{功能需求}
    \subsection{xxx1.1 作业/实验查询}
      \subsubsection{介绍}
	 作业/实验查询用于查询作业/实验情况。学生可以查到本人相关课程的作业情况。教师可以查询相应班级所有学生每次作业的提交情况。教学管理人员可以查询到自己所负责所有学生相应的作业提交情况。
	 \subsubsection{输入}
    	 不同用户的输入如下:
    	 \begin{center}\begin{description}
         \item[学生] 学生的输入为选择响应的课程或总体...
         \item[老师] 老师可以选择相应的班级和某次作业或全部作业...
         \item[教秘] 教秘可以可以选择自己所负责范围内的所有学生和课程的相应信息
    	 \end{description}\end{center}
	 \subsubsection{处理}
	 A. 连接服务器,检查查询相应的权限,查询数据库。

	 B. 连接查询失败,提示超时,并提示用户检查网络或重试。

	 C. 连接查询成功,更新本地数据。

	 \subsubsection{输出}
	 A. 显示更新后的本地数据或显示从服务器返回的数据。

	 B. 学生端更新相应的日程。

    \subsection{xxx1.2 作业/实验/通知发布}
      \subsubsection{介绍}
 	   作业/实验/通知发布仅供教师端和管理人员使用。可以用于教师发布自己负责课程相关的作业、实验或通知。
	 \subsubsection{输入}
	   发布首先要新建条目(某次作业或通知等),之后输入相应发布的信息。
	   输入数据内容如下:
    	   \begin{itemize}
	   \item 作业/实验/通知的名称。
        \item 作业/实验/通知的文本内容。
        \item 学生提交的截至日期,以日历形式选择。
        \item 相应作业的材料,选择上传按键,上传相应文件。
        \end{itemize}
	 \subsubsection{处理}
	 A. 连接服务器,检查查询相应的权限。

	 B. 连接失败或权限不对,提示用户。

	 C. 连接成功且权限正确,将数据更新到后台数据库中,将文件保存在服务器中。

	 \subsubsection{输出}
	 相关数据存入服务器,客户端提示发布成功或其他异常信息。
    \subsection{xxx.3 作业/实验/通知修改删除}
      \subsubsection{介绍}
	 仅供教师和管理人员使用。用于修改或删除已发布的作业/实验/通知。
	 \subsubsection{输入}
	 输入为准备修改的作业等的名称。进入发布界面后,输入要修改的内容或选择删除该作业、通知等。
	 \subsubsection{处理}
	 A. 连接服务器,检查查询相应的权限。

	 B. 连接失败或权限不对,提示用户。

	 C. 连接成功且权限正确,将数据更新到后台数据库中,将文件保存在服务器中。

	 \subsubsection{输出}
	 在客户端上显示更新后的结果。
    \subsection{xxx.4 作业/实验/通知提交}
      \subsubsection{介绍}
	 仅供学生使用。学生提交完成的作业、实验或需要提交材料的通知。
	 \subsubsection{输入}
	 首先学生选择本人准备提交的作业。提交的内容可以如下:
	 \begin{itemize}
        \item 作业/实验/通知的文本内容。
        \item 相应的材料,选择上传按键,上传相应文件。
      \end{itemize}
	 \subsubsection{处理}
	 A. 连接服务器

	 B. 连接失败,提示超时,并提示用户检查网络或重试。
.
	 C. 若连接成功,检测当前日期与该作业截止日期。

	 D. 若已过作业截止日期,则提交失败。

	 E. 若未到作业截至时间,上传数据到服务器,返回状态。

	 F. 从服务器返回成功,提示上传成功。

	 G. 从服务器返回失败,提示用户上传失败。

	 \subsubsection{输出}
	 A. 上传成功与否的提示

	 B. 上传成功则作业完成状况标示为已完成,并更新相应的日程。

	\subsection{xxx.5 作业/实验查看与批改}
      \subsubsection{介绍}
	 仅供教师端使用。作业/实验查看与批改用与教师查看学生相应的作业或实验。
	 \subsubsection{输入}
	 A. 教师选择相应的作业,可选择将本次作业全部下载到本地或在线查看。

	 B. 教师选择要查看/批改作业学生的名称或学号。

	 C. 教师输入该同学的作业成绩。

	 输入成绩数据要求如下:
      \begin{itemize}
       \item 成绩为整数,如果含有小数按四舍五入进行舍入。
       \item 默认成绩的范围为[0,100],超过此范围进行警告并且自动取最接近的可行值。可以通过自定义设置进行调整。
     \end{itemize}
	\subsubsection{处理}
	 A. 检查相应的权限。

	 B. 将相应数据发送至客户端。

	 C. 通过按照默认规则和用户自定义的规则生成相关的约束对于输入的成绩的有效性进行验证,检测成功后更新数据库。

	\subsubsection{输出}
	A. 客户端可看到学生的作业,成绩同步到数据库。

	B. 该生的本次作业情况更新为已批改。

	C. 显示待批改作业份数。











