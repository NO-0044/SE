\chapter{软件质量特性}
	\section{适应性 可用性}
\begin{itemize}
	\item 本软件的ios版本以适配最新的ios版本为目标,并测试出在其它ios版本上的运行情况,在会出现问题的ios版本上禁止运行本软件,并提示更新系统版本。

	\item 本软件的Android版本以适配最新的Android版本为目标,并测试出在其它版本上的运行情况,在会出现问题的Android版本上禁止运行本软件,并提示更新系统版本。
\end{itemize}

	\section{正确性}
	本软件在发布前,进行详细的单元测试,以保证每一部分正确。

	在发布前使用实际数据进行整体性测试,并进行用户测试,以保证整个软件的正确性。

	\section{灵活性}
	在保证正确性的同时,尽量提高灵活性。

	本软件用模块化的设计,使增加删除功能变得容易。

	\section{可维护性}
	采用模块化设计,结构合理。同时,定义接口清晰,文档完整清楚,保证良好的可维护性。

	\section{易学性与易用性}
	本软件要界面清晰简洁,功能设计排布合理。必须要有较强的易用性,在保证易用性的前提下增加软件的易学性。

	\section{可靠性}
	本软件的可靠性包括,在服务器上保存的数据不会丢失,在客户端上用户编辑中的数据不会在本地丢失,软件不会突然崩溃。这要求本软件的服务器端有数据异地容灾备份,并且保证用户获得的是最新版本的数据。
