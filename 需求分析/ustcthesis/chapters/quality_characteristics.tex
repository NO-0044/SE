\chapter{软件质量特性}
<Specify any additional quality characteristics for the project that will be important to either the customers or the developers. Some to consider are: adaptability, availability, correctness, flexibility, interoperability, maintainability, portability, reliability, reusability, robustness, testability, and usability. Write these to be specific, quantitative, and verifiable when possible. At the least, clarify the relative preferences for various attributes, such as ease of use over ease of learning.

详细说明项目任何其他的质量特性。该特性对客户和开发者都非常重要。考虑的方面包括:适应性,可用性,正确性,灵活性,交互工作能力,可维护性,可移植性,可靠性,可重用性,鲁棒性,可测试性和可用性等。定量的详细描述这些特性,尽可能的可验证。对不同属性之间的重要性加以阐述,如:易用性比易学性更重要。

<Please use the below sub-section for each attributes separately. You can copy the section for additional attributes. >

每一个属性单独使用一个小节描述,可根据需要进行增减,如增加可维护性小节等。

	\section{适应性 可用性}
	本软件的ios版本以适配最新的ios版本为目标,并测试出在其它ios版本上的运行情况,在会出现问题的ios版本上禁止运行本软件,并提示更新系统版本。

	本软件的Android版本以适配最新的Android版本为目标,并测试出在其它版本上的运行情况,在会出现问题的Android版本上禁止运行本软件,并提示更新系统版本。 

	本软件的Web版本以Chrome为测试平台。
	\section{正确性}
	本软件在发布前,进行详细的单元测试,以保证每一部分正确。

	在发布前使用实际数据进行整体性测试,并进行用户测试,以保证整个软件的正确性。

	\section{灵活性}
	在保证正确性的同时,尽量提高灵活性。

	本软件用模块化的设计,使增加删除功能变得容易。
	
	\section{可维护性}
	采用模块化设计,结构合理。同时,定义接口清晰,文档完整清楚,保证良好的可维护性。
	
	\section{易学性与易用性}
	本软件要界面清晰简洁,功能设计排布合理。必须要有较强的易用性,在保证易用性的前提下增加软件的易学性。 

	\section{可靠性}
	本软件的可靠性包括,在服务器上保存的数据不会丢失,在客户端上用户编辑中的数据不会在本地丢失,软件不会突然崩溃。这要求本软件的服务器端有数据异地容灾备份,并且保证用户获得的是最新版本的数据。