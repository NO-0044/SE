\chapter{数据结构设计}
\section{逻辑结构设计}
  \subsection{服务器端数据结构}
    \subsubsection{客户端请求}
    请求以socket的形式以字节流的格式传输到服务器端口。
    \begin{lstlisting}[language=C,caption=客户端请求]
      struct command{
      Byte b;
      }
    \end{lstlisting}
    \subsubsection{登录请求}
    解析结果是登录请求,请求字段分别为用户id和密码,两个都加密后进行传输。
    \begin{lstlisting}[language=C,caption=客户端请求]
      struct LoginCommand{
      commandID id;
      String userid;
      String password;
      }
    \end{lstlisting}
    \subsubsection{作业/实验/通知查询请求}
    解析结果为作业实验查询请求,请求字段为用户id和请求查询的作业/实验名称。
    \begin{lstlisting}[language=C,caption=作业/实验/通知查询请求]
      struct WorkQueryCommand{
      commandID id;
      String userid;
      String workid;
      }
    \end{lstlisting}
    \subsubsection{作业/实验/通知发布请求}
    向服务器发布作业/实验/通知。解析为发布请求,请求字段为用户名,之后为字节流,内容是作业/实验/通知的详细信息,具体解析工作交由作业/实验/通知功能结构完成。
    \begin{lstlisting}[language=C,caption=作业/实验/通知发布请求]
      struct WorkPublishCommand{
      commandID id;
      String userid;
      Byte s;
      }
    \end{lstlisting}
    \subsubsection{作业/实验/通知修改请求}
    修改已经发布的作业/实验/通知。解析为修改请求,请求字段为用户名以及需要修改的工作名,之后为字节流,内容为需要进行修改的详细内容,具体解析工作交由作业/实验/通知功能结构完成。
    \begin{lstlisting}[language=C,caption=作业/实验/通知修改请求]
      struct WorkAdjustCommand{
      commandID id;
      String userid;
      String workid;
      Byte s;
      }
    \end{lstlisting}
    \subsubsection{作业/实验提交请求}
    提交所完成的作业/实验。解析为提交请求,请求字段为用户名以及待提交的作业,之后为字节流,内容为需要提交的具体成果,具体解析工作交由作业/实验/通知功能结构完成。
    \begin{lstlisting}[language=C,caption=作业/实验提交请求]
      struct WorkCommitCommand{
      commandID id;
      String userid;
      String workid;
      Byte s;
      }
    \end{lstlisting}
    \subsubsection{作业/实验查看请求}
    查看所完成的作业/实验。解析为提交请求,请求字段为用户名以及待查看的作业。
    \begin{lstlisting}[language=C,caption=作业/实验查看请求]
      struct WorkCheckCommand{
      commandID id;
      String userid;
      String workid;
      }
    \end{lstlisting}
    \subsubsection{成绩录入请求}
    解析结果是录入成绩请求,请求字段是用户id,作业id,之后是字节流,内容为具体学生和成绩的信息,具体解析交由成绩录入功能模块处理。
    \begin{lstlisting}[language=C,caption=成绩录入请求]
      struct ScoreCommitCommand{
      commandID id;
      String userid;
      String workid;
      Byte s;
      }
    \end{lstlisting}
    \subsubsection{成绩查询请求}
    解析结果是录入成绩请求,请求字段是用户id,待查询用户id,作业id。
    \begin{lstlisting}[language=C,caption=成绩查询请求]
      struct ScoreCommitCommand{
      commandID id;
      String userid;
      String queryid;
      String workid;
      }
    \end{lstlisting}
    \subsubsection{成绩统计调整请求}
    解析结果是成绩统计调整请求,请求字段是用户id,作业id,后面跟有字节流,内容为具体的更改或者统计信息。
    \begin{lstlisting}[language=C,caption=成绩统计调整请求]
      struct ScoreAdjustCommand{
      commandID id;
      String userid;
      String workid;
      Byte s;
      }
    \end{lstlisting}
    \subsubsection{资源上传请求}
    解析结果是资源上传请求,请求字段是用户id,后面跟有字节流,内容为上传的目的地址,上传文件的相关信息和具体资源数据。具体解析交由资源上传模块解析。
    \begin{lstlisting}[language=C,caption=资源上传请求]
      struct ResoureUploadCommand{
      commandID id;
      String userid;
      Byte s;
      }
    \end{lstlisting}
    \subsubsection{资源下载请求}
    解析结果是资源下载请求,请求字段是用户id和资源id。
    \begin{lstlisting}[language=C,caption=资源下载请求]
      struct ResourceDownloadCommand{
      commandID id;
      String userid;
      String resourceid;
      }
    \end{lstlisting}
    \subsubsection{日程同步请求}
    解析结果是日程同步请求,请求字段是用户id,后面跟有字节流,内容是本地等待同步的日程信息。具体解析交由日程同步模块处理。
    \begin{lstlisting}[language=C,caption=日程同步请求]
      struct ScheduleSynCommand{
      commandID id;
      String userid;
      Byte s;
      }
    \end{lstlisting}
    \subsubsection{讨论区新建/回复请求}
    解析结果是讨论区相关请求,请求字段是用户id,后面跟有字节流,内容为具体的主题/回复信息。具体解析交由讨论区模块处理。
    \begin{lstlisting}[language=C,caption=讨论区新建/回复请求]
      struct NewDisCommand{
      commandID id;
      String userid;
      Byte s;
      }
    \end{lstlisting}
    \subsubsection{讨论区查询请求}
    解析结果是讨论区查询请求,请求字段是用户id和需要查询的主题。
    \begin{lstlisting}[language=C,caption=讨论区查询请求]
      struct DisQueryCommand{
      commandID id;
      String userid;
      String thesis
      }
    \end{lstlisting}
    \subsubsection{新建博文请求}
    解析结果是新建博文请求,请求字段是用户id,后面跟有字节流,内容为具体博文内容。具体解析交由博文模块处理。
    \begin{lstlisting}[language=C,caption=新建博文请求]
      struct NewBlogCommand{
      commandID id;
      String userid;
      Byte s;
      }
    \end{lstlisting}
    \subsubsection{学习笔记同步请求}
    解析结果是学习笔记同步请求,请求字段是用户id,后面跟有字节流,内容为本地待提交的笔记内容,具体交由学习笔记同步模块解析。
    \begin{lstlisting}[language=C,caption=学习笔记同步请求]
      struct NoteSynCommand{
      commandID id;
      String userid;
      Byte s;
      }
    \end{lstlisting}
  \subsection{客户端数据结构}
  \subsubsection{作业/实验信息}
  \begin{lstlisting}[language=C,caption=作业/实验信息]
    struct WorkInfo{
      String caption;
      String content;
      Data dedline;
      String[] userid;
      String publisher;
    }
  \end{lstlisting}
  作业实验的信息包括标题、详细内容、提交截止时间、需要提交作业的用户以及发布者。
  \subsubsection{通知信息}
  \begin{lstlisting}[language=C,caption=通知信息]
    struct NoticeInfo{
      String caption;
      String content;
      String[] userid;
      String publisher;
    }
  \end{lstlisting}
  通知信息包括标题,详细内容,需要接受消息的用户以及消息的发布者。
  \subsubsection{个人成绩}
  \begin{lstlisting}[language=C,caption=个人成绩]
    struct ScoreEntry{
      int score;
      String userid
    }
  \end{lstlisting}
  每条成绩记录包括用户和其相应的成绩。
  \subsubsection{成绩表}
  \begin{lstlisting}[language=C,caption=成绩表]
    struct ScoreList{
      String Test;
      ScoreEntry[] score;
    }
  \end{lstlisting}
  成绩记录中包括考试/作业名称和各个用户的成绩。
  \subsubsection{资源信息}
  \begin{lstlisting}[language=C,caption=资源信息]
    struct ResourceInfo{
      String name;
      int type;
      int size;
      Byte content;
    }
  \end{lstlisting}
  资源信息包括资源名称,资源的类型,资源大小,资源内容等等。
  \subsubsection{日程信息}
  \begin{lstlisting}[language=C,caption=日程信息]
    struct ScheduleInfo{
      Data data;
      int duration;
      String name;
      String place;
      String description;
    }
  \end{lstlisting}
  每条日程包括日程的时间地点,事件的持续时间,事件的名称和详细描述。
  \subsubsection{讨论区主题}
  \begin{lstlisting}[language=C,caption=讨论区主题]
    struct DisThesis{
      String title;
      String content;
      Data creationtime;
      Vector<ReplyInfo> reply;
      String author;
    }
  \end{lstlisting}
  每个讨论区主题包含创建时间、标题、内容、作者以及回复信息。
  \subsubsection{讨论区回复}
  \begin{lstlisting}[language=C,caption=讨论区回复]
    struct ReplyInfo{
      String content;
      String author;
      Data time;
    }
  \end{lstlisting}
  每条回复包括回复内容、回复者和时间信息。
  \subsubsection{博文信息}
  \begin{lstlisting}[language=C,caption=博文信息]
    struct BlogInfo{
    String title;
    String author;
    Data time;
    int type;
    }
  \end{lstlisting}
  每条博文记录包括具体内容、作者、时间以及是否公开。
  \subsubsection{学习笔记}
  \begin{lstlisting}[language=C,caption=学习笔记]
    struct NoteData{
      String name;
      Data time;
      String author;
      String content;
    }
  \end{lstlisting}
  每个学习笔记包括标题、内容、创建时间和作者信息。








  \subsection{用户端数据结构}

\section{物理结构设计}
各个数据结构无特殊物物理结构要求。

\section{数据结构与程序模块的关系}
[此处指的是不同的数据结构分配到哪些模块去实现。可按不同的端拆分此表]
\begin{table}[htbp]
\centering
\caption{数据结构与程序代码的关系表} \label{tab:datastructure-module}
\begin{tabular}{|c|c|c|c|}
    \hline
    · & 模块1 & 模块2 & 模块3 \\
    \hline
    结构1 & · & Y & · \\
    \hline
    结构2 & · & Y & · \\
    \hline
    结构3 & · & Y & · \\
    \hline
    结构4 & Y & · & · \\
    \hline
    结构5 & · & · & Y \\
    \hline
\end{tabular}
\note{各项数据结构的实现与各个程序模块的分配关系}
\end{table}
