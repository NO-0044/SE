\chapter{引言}
\section{编写目的}
在本项目的前一阶段,也就是需求分析阶段,已经将系统用户对本系统的需求做了详细的阐述,这些用户需求已经在上一阶段中对不同用户所提出的不同功能,实现的各种效果做了调研工作,并在需求规格说明书中得到详尽得叙述及阐明。

本阶段在系统的需求分析的基础上,对学习/教学辅助软件系统做概要设计。主要解决了实现该系统需求的程序模块设计问题。包括如何把该系统划分成若干个模块、决定各个模块之间的接口、模块之间传递的信息,以及数据结构、模块结构的设计等。在以下的概要设计报告中将对在本阶段中对系统所做的所有概要设计进行详细的说明,在设计过程中起到了提纲挈领的作用。

在下一阶段的详细设计中,程序设计员可参考此概要设计报告,在概要设计所做的模块结构设计的基础上,对系统进行详细设计。在以后的软件测试以及软件维护阶段也可参考此说明书,以便于了解在概要设计过程中所完成的各模块设计结构,或在修改时找出在本阶段设计的不足或错误。


\section{项目背景}
为了能更加高效的开展教学工作,充分利用各类信息化工具,市场上出现了各类教学辅助平台和学习辅助工具,因为其便捷和高效受到人们的青睐。在广泛的应用之中也的确方便了广大学生和教职人员。\\
但是仍然普遍存在以下这些问题:\\
\begin{itemize}
  \item 各产品功能分散不完善,往往需要同时交错使用多个系统
  \item 功能落后,更新不及时
  \item 用户界面不美观,用户体验差
  \item 处理不够智能,系统内部模块协同差,使用不便捷
\end{itemize}
为此,我们提出了此项目,希望能够整合现有各系统的各项功能并在其基础之上加强模块协同,融合新的设计理念,加强用户体验。最终能够更好的服务于广大教师的教学工作并能够为同学的学习提供辅助,提高学习效率。


\section{术语}
[列出本文档中所用到的专门术语的定义和外文缩写的原词组]
\begin{table}[htbp]
\centering
\caption{术语表} \label{tab:terminology}
\begin{tabular}{|c|c|}
    \hline
    缩写、术语 & 解释 \\
    \hline
    c & d \\
    \hline
\end{tabular}
% \note{这里是表的注释}
\end{table}
